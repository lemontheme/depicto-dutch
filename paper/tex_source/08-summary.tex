%!TEX root = depicto-top.tex
\addcontentsline{toc}{chapter}{Summary}
\chapter*{Summary}
% max. 3000 chars ≈ 450 words.

In recent years, increasing attention has been paid to the potential of using
pictographs to open up the online world, so that users with intellectual
disabilities can benefit from the same tools for remote communication (email,
instant messaging, social media) which define so much of what it means to be a
socially active member of society nowadays. This thesis describes the
development of an automatic translation system that aims to enable
language-impaired, intellectually disabled individuals to compose written
messages simply by selecting a sequence of pictographic images.

By way of contrast with existing approaches for pictograph-to-text translation,
the system that we develop here, \depicto, takes a 100% rule-based
approach. That is, all stages in the translation process make use of linguistic
rules, as opposed to statistical data. On paper, such an approach has several
advantages over data-driven alternatives. In particular, it makes it possible
to encode elegant generalizations about the pictographic input, such that
translation is maximally expressive, the output is consistent, and one of the
main challenges for statistical methods is overcome, i.e., that there is no
data on which to train a model of the pictographic input. Rule-based approaches
are also costly, though, requiring resources developed in large part by hand.
Thus, aside from the obvious objective of testing whether this approach can be
realized (i.e., by getting \depicto\ to work), we also explore how development
can be made more \emph{feasible}. In addition, we set two design criteria:
first, the system must be \emph{sensitive} to the needs of its users; second,
it must be possible to \emph{extend} to other target languages. (Currently, the
system translates to Dutch.)

In \cref{chap:Background}, we introduce all third-party resources used by the
system. In \cref{chap:pipeline1} we show how the pictographic symbol set
\sclera\ can be analysed as a natural language and how this language can be
modelled by a constraint-based grammar. This grammar is written in an
implemented variant of the \hpsg\ framework and forms the core of the first of
\depicto's three modules: the analysis module. In the first half of
\cref{chap:pipelineii}, we see the semantic structure of analysed pictographic
sequences is converted, or \emph{transferred}, so as to be compatible with the
input expected by the target language grammar. This happens in the second
module in the \depicto\ chain. In the second half of \cref{chap:pipelineii}, we
describe a basic grammar model of Dutch and show how this is used to
\emph{generate} well-formed sentences based on the translated semantic
structure yielded by the second module. Next, in the first half of
\cref{chap:Conclusion}, we evaluate the system as a whole. We argue that both
its precision, i.e., ability to produce well-formed output, and performance are
high, but are forced to concede that its coverage is limited. We also show how
the \depicto\ system fares when pitted against a fundamentally statistical
system, namely, the \emph{Picto2Text} system (\citep{sevens2015natural}).

All in all, we conclude that \depicto\ is able to successfully translate (a
very limited subset of) pictographic sequences into well-formed natural
language sentences. Moreover, by adopting an explicitly modular design, we try
to make the system as appealing as possible to developers dealing with other
target languages. At the same time, while modularity certainly helps, extending
\depicto's analysis module remains a costly task. Whether such costliness is an
adequate trade-off for the indisputably high quality of the system's output
will be determined by future work. As a \emph{translation system}, \depicto\
can be considered a success. As an \emph{assistive writing tool}, however, it
needs improvement. Its analysis module imposes overly stringent constraints on
the order of elements on the pictographic input, and its generation module is
set to output every single sentence that is well-formed according to the
grammar of the target language. In future work, we will focus on minimizing the
first limitation and removing the second altogether. This will involve exciting
experiments in hybridizing \depicto's rule-based core with other, more
statistical approaches to translation/generation.
