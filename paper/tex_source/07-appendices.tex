%!TEX root = depicto-top.tex
    % \begin{appendices}

\appendix
\chapter{Demoing \depicto}
\label{app:settingupdepicto}

This appendix provides instructions for obtaining and using the version of the
\depicto\ system developed in this thesis. This version takes a relatively
small set of \sclera\ pictograph identifiers as input and produces well-formed
\textbf{Dutch} sentences as output. For a complete list of currently accepted
pictograph identifiers, please see Appendix~\ref{app:sclera-lexicon}.

\section*{Requirements}

The \depicto\ system supports 64-bit versions of both GNU/Linux and Mac OS X.
It has been tested on Ubuntu (verions 15.10 and 16.04.1) and on Mac OS X
10.11.5 (El Capitan).

\section*{Setting up}

The grammars used by the \ace\ processor need to be compiled from source. The
source code of the individual grammars, as well as up-to-date \ace\ processor
binaries (for each of the two supported OSs), a shell script wrapper, and some
very preliminary documentation, is available as a GitHub repository.

\subsection*{Getting the code}

To obtain the repository, \textbf{either} ...

\begin{enumerate}
    \item Visit the main page of the repository at
          \url{https://github.com/lemontheme/depicto-dutch},
    \item Click \textbf{Clone or download} >
          \textbf{Download Zip}
    \item Extract the Zip archive
\end{enumerate}

\textbf{or}, if Git is installed (preferred method), ...

\begin{enumerate}
    \item Open Terminal
    \item Change the current working directory (\texttt{cd}) to the location
          where you want the repository to be downloaded,
    \item Type:
    \begin{verbatim}
$ git clone https://github.com/lemontheme/depicto-dutch
    \end{verbatim}
    \vspace{-0.5cm}
    \item Press \textbf{Enter}.
\end{enumerate}

\subsection*{Compiling the grammars}

The most important directories in the downloaded repository are
\texttt{ace-app} and \texttt{grammars}. The \texttt{ace-app} directory contains
OS-specific \ace\ binaries, shell script wrappers, and compiled grammar images
(as \texttt{.dat} files). The \texttt{grammars} directory contains those files
which constitute the `code' (\texttt{.tdl} files) from which the individual
grammars are compiled. The subdirectories under \texttt{grammars} are as
follows:

\begin{itemize}
    \item \texttt{mini-dutch}  = Grammar of Dutch
    \item \texttt{mini-sclera} = Grammar of \sclera
    \item \texttt{matrix-main} = Resources shared by both
                                 \texttt{mini-dutch} and \texttt{mini-sclera}
    \item \texttt{transfer-nl} = Sclera-Dutch transfer grammar
\end{itemize}

To compile the grammars, follow the steps below.

\begin{enumerate}
    \item In Terminal, navigate to the subdirectory of \texttt{ace-app/}
          whose name corresponds to that of the current operating system, e.g.:
  \begin{verbatim}
$ cd depicto-dutch/ace-app/osx #(on OS X)
  \end{verbatim}
  \vspace{-0.5cm}
    or
  \begin{verbatim}
$ cd depicto-dutch/ace-app/linux #(on Linux)
  \end{verbatim}
  \vspace{-0.5cm}
    \item Run the \texttt{compile-all} shell script:
  \begin{verbatim}
$ sh compile-all_osx.sh
  \end{verbatim}
  \vspace{-0.5cm}
    or
  \begin{verbatim}
$ sh compile-all_linux.sh
  \end{verbatim}
  \vspace{-0.5cm}
  (\texttt{compile-all_{}.sh} is a simple wrapper containing three individual
   compilation calls to \ace\ processor.)
\end{enumerate}

\section*{Usage}

For ease of testing, the \depicto\ demo includes a second wrapper that calls
all three stages of the pipeline in order, invoking \ace\ plus an appropriate
grammar image for each, and `piping' the output of preceding stages forward.
This wrapper, which is broken down in \cref{ex:conc:eval:time}, represents only
one of many ways in which the \depicto\ can be implemented. For instance, while
it relies on interactive user input itself, a slightly modified script could
just as easily read input from a test file, or from any other kind of input
stream.

\begin{enumerate}
    \item When starting a new session, open Terminal and
          \texttt{cd} to the subdirectory of \texttt{ace-app/}
          whose name corresponds to that of the current operating system.
    \item Initialize the \depicto\ pipeline, with...
  \begin{verbatim}
$ sh test-pipeline_osx.sh
  \end{verbatim}
  \vspace{-0.5cm}
    or
  \begin{verbatim}
$ sh test-pipeline_linux.sh
  \end{verbatim}
  \vspace{-0.5cm}
        This opens up new line below,
        from which \depicto\ (or the \ace\ processor, to be precise)
        waits to read its input.
  \item Enter a sequence of pictograph identifiers, seperated by any
        number of spaces, and press \textbf{Enter}. For example:
  \begin{verbatim}
$ sh test-pipeline_linux.sh
dog see bus yesterday #<Enter>
  \end{verbatim}
  \vspace{-0.5cm}
        If the input is covered by the system, this step results in
        a complete list of translation hypotheses, as described in \cref{subs:generation}.

        Whether successful or not, the pipeline returns a new
        user input line, so that this step can be repeated infinitely.

    \item To exit the pipeline, type \texttt{<CTRL-C>}.

\end{enumerate}

% \subsection{Note to Linux users in particular!}

% export TMPDIR=$HOME/tmp

% \appendix
% \vspace{-5cm}
\chapter{\sclera-grammar lexicon}
\label{app:sclera-lexicon}

\begin{figure}[h!]
\centering
\small
\begin{tabular}{ | p{3.5cm} | p{3.5cm} | p{3.5cm} | p{3.5cm} | }
    \hline
    \textbf{Picto identifier} & \textbf{Approx. PoS} & \textbf{Main relation} & \textbf{Other relation(s)} \\
    \hline
    I          & (pro)\textbf{n}oun (nom/acc) &   \texttt{`\_pronoun\_n\_rel'}       &        \\
    you\_sg        & (pro)\textbf{n}oun (nom/acc) &    \texttt{`\_pronoun\_n\_rel'}      &        \\
    you\_pl        & (pro)\textbf{n}oun (nom/acc)  &   \texttt{`\_pronoun\_n\_rel'}       &        \\
    he         & (pro)\textbf{n}oun (nom/acc) & \texttt{`\_pronoun\_n\_rel'}         &        \\
    she        & (pro)\textbf{n}oun (nom/acc)  & \texttt{`\_pronoun\_n\_rel'}         &        \\
    we         & (pro)\textbf{n}oun (nom/acc) & \texttt{`\_pronoun\_n\_rel'}         &        \\
    they       & (pro)\textbf{n}oun (nom/acc)  & \texttt{`\_pronoun\_n\_rel'}         &        \\
    dog        & count \textbf{n}oun & \texttt{`\_dog\_n\_rel'}         &        \\
    girl       & count \textbf{n}oun & \texttt{`\_girl\_n\_rel'}         &        \\
    bus        & count \textbf{n}oun & \texttt{`\_bus\_n\_rel'}         &        \\
    kiss       & count \textbf{n}oun & \texttt{`\_kiss\_n\_rel'}         &        \\
    love       & mass \textbf{n}oun & \texttt{`\_love\_n\_rel'}         &        \\
    couch      & count \textbf{n}oun & \texttt{`\_couch\_n\_rel'}         &        \\
    headphones & plural \textbf{n}oun & \texttt{`\_headphones\_n\_rel'}         &        \\
    happy      & ad\textbf{j}ective     & \texttt{`\_happy\_j\_rel'}         &        \\
    brown      & ad\textbf{j}ective     & \texttt{`\_brown\_j\_rel'}         &        \\
    yesterday  & temp. adve\textbf{r}b     & \texttt{`\_yesterday\_r\_rel'}         &        \\
    sleep      & \textbf{v}erb     & \texttt{`\_sleep\_v\_rel'}         &        \\
    bark       & \textbf{v}erb     & \texttt{`\_bark\_v\_rel'}         &        \\
    buy        & \textbf{v}erb     & \texttt{`\_buy\_v\_rel'}         &        \\
    see        & \textbf{v}erb     & \texttt{`\_see\_v\_rel'}         &        \\
    give       & \textbf{v}erb     & \texttt{`\_give\_v\_rel'}         &        \\
    brush      & \textbf{v}erb     &         \texttt{`\_brush\_v\_rel'} &        \\
    dog\_bark  & complex-verb & \texttt{`\_bark\_v\_rel'} & \texttt{`\_dog\_n\_rel'}, \texttt{`q\_rel\_min'}          \\
    brush\_dog & complex-verb & \texttt{`\_brush\_v\_rel'} & \texttt{`\_dog\_n\_rel'}, \texttt{`q\_rel\_min'}      \\
    go\_school & complex-verb & \texttt{`\_go\_v\_rel'}   & \texttt{`locative\_rel'}, \texttt{`\_school\_v\_rel'}, \texttt{`exist\_q\_rel'} \\
    question   & illoc. `operator' &  (none)   &   \\
    \hline
\end{tabular}

\end{figure}

%  picto identii
%
%
%


\chapter{Example test sentences}

\begin{verbatim}

dog see bus
dog see bus yesterday
dog see bus yesterday question
brown dog see bus yesterday question
dog sleep
dog_bark
brown dog_bark
i buy headphones
i give you_sg headphones
girl go_school
she go_school yesterday
happy dog_bark
happy girl buy brown dog
i brown brush_dog
i brown brush_dog yesterday
you_sg give i kiss question
they see i yesterday
i see you_pl yesterday

\end{verbatim}

% \end{appendices}
